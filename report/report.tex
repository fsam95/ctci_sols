\documentclass{article}

\title{Cracking the Coding Interview}
\begin{document}

\section*{Preface}
I thought it would be a good idea to document my endeavour 
to complete Cracking the Coding Interview. It would be good
for me to be able to articulate my approaches and maybe it would help you to 
see how I arrived at (or didn't arrive) at some of the solutions. \\

I was also driven to write this because I'm a little concerned my writing
skills with begin atrophying since I finished my undergraduate studies.
\section*{Recursion}
\subsection*{Problem 8.6}
I couldn't figure out this problem on my own after thinking for about 40 minutes.
I thought about creating a tree rooted at 0 with each branch representing the
addition of one of the four denominations (cent, nickel, dime, quarter). I
would then count all of the nodes that have the value $n$ and that would be my result. \\

However, this would lead to counting permutations of the same change denominations (e.g.
a nickel and 5 cents would be included as change for a dime but 5 cents and a nickel
would also be counted even though its the same combination as the former). \\

Even the proposed solution was confusing to me because it wasn't clear why returning 1 
for the penny case accounts for the accumulation of all the different ways of calculating
the change. It becomes more clear if you draw a picture of what exactly is being counted.






\end{document}
